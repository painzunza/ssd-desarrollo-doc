\section{Preparación Ambiente}
\label{sec:2-ambiente}

  Para utilizar los servicios diseñados e implementados
  usando el sistema \ssdAxis\ 
  es necesario contar con una serie de prerrequisitos
  a instalar en el sistema base donde van a operar
  los servicios.
  Las características del sistema base
  en las que se implementa la estructura de servicios
  para este trabajo
  es resumida en la tabla \ref{tab:2-1-sistema-base}.

  \begin{table}[H]
    \centering
    \begin{tabular}{ll}
      \hline
      Sistema Operativo & Fedora 22 WorkStation \\
      \hline
      Placa Madre       & ASUSTeK M4A78LT-M-LE \\
      \hline
      Procesador        & AMD Athlon\texttrademark\ II X2 250 (dual core) 64 bits \\
      \hline
      Memoria           & 2 \(\times\)  DIMM Synchronous \(1333 [MHz]\) \(2 [GiB]\) \\
      \hline
    \end{tabular}
    \caption{Características sistema base}
    \label{tab:2-1-sistema-base}
  \end{table}

  A continuación se procede a describir
  cada programa y/o paquete de software
  instalado en el sistema base
  para la implementación y operación
  de los servicios diseñados.

\subsection{Lenguaje Base}
\label{subsec:2-1-lenguaje-base}

  El sistema provisto por \ssdAxis\
  es compatible con los lenguajes de programación
  \emph{Ansi C} y \emph{Java}.
  Para ambos lenguajes existen librerías que permiten
  programar desde cero los clientes
  \emph{SOAP} para el experimento,
  sin embargo el lenguaje \emph{Java}
  posee la ventaja de ser el lenguaje nativo
  del sistema \ssdAxis.
  Adicionalmente,
  el servidor web más popular para
  levantar servicios y aplicaciones
  programadas en \emph{Java}
  llamado \ssdTomcat\ 
  es altamente compatible con
  el sistema \ssdAxis\ y sus predecesores.

  \paragraph{Java}
  Para programar y ejecutar aplicaciones
  escritas usando el lenguaje \emph{Java}
  es necesario instalar el
  sistema de desarrollo provisto por 
  los mantenedores del sistema.
  Al año 2015,
  tanto el lenguaje \emph{Java}
  como todos los servicios asociados
  creados por la empresa
  \ssdSun\ son propiedad de
  la empresa \ssdOracle\ 
  luego de que la primera adquiriera a la última
  durante el año 2010.
  Actualmente
  \ssdOracle\ mantiene una versión propietaria
  del lenguaje y su ambiente de ejecución
  conocido como la \emph{Java Virtual Machine}
  o \emph{JVM},
  sin embargo existen versiones \emph{open source}
  disponibles en la mayoría de las distribuciones
  de \emph{Linux} apoyadas por \ssdOracle.
  Para este trabajo se describirá como instalar
  el ambiente de desarrollo 
  \emph{open JDK} para el sistema base a utilizar.

  \paragraph{Java Development Kit}
  Para el sistema operativo \emph{Fedora 22}
  hay varias formas de instalar el 
  ambiente de desarrollo.
  En líneas generales se necesitan
  tres componentes para utilizar el lenguaje
  \emph{Java}:
  \begin{itemize}
  \item[\textbf{CI}]
    \emph{Módulos.}
    Que contienen los servicios básicos
    o estándar del sistema,
    así como las librerías para
    realizar tareas más avanzadas
    como programación distribuída en la red
    o programación en paralelo con múltiples
    hilos de ejecución.
  \item[\textbf{CII}]
    \emph{Compilador Bytecode.}
    Que traduce el código fuente escrito
    en \emph{Java} a un dialecto
    ejecutable por la \emph{JVM}.
  \item[\textbf{CIII}]
    \emph{Runtime Environment.}
    Correspondiente a la \emph{JVM}
    implementada y programas de apoyo,
    como opciones de depuración y perfil.
  \end{itemize}
  Estas tres componentes vienen incluídas
  en varios paquetes de distribución publicados
  en Internet.
  La versión \emph{open source}
  a usar corresponde al 
  \emph{Open Java Development Kit}
  u \emph{Open JDK}\footnote{\ssdJDKLink}.
  
  \paragraph{Instalación}
  Para instalar el paquete 
  \emph{Open JDK}
  en el sistema base se puede seguir
  dos cursos de acción,
  donde el primero corresponde a descargar
  el código fuente completo
  del \emph{JDK}
  y proceder a construirlo en el sistema base.
  Para esto es necesario descargar la última versión
  desde el repositorio \emph{Mercurial} oficial 
  \url{http://hg.openjdk.java.net/jdk8/jdk8/jdk}
  

  